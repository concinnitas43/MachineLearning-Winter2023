%!TeX program=pdflatex
\documentclass[]{sptr_article}

\usepackage[finemath]{kotex}
\usepackage{dhucs-nanumfont}

\usepackage{biblatex}
\addbibresource{ref.bib}

\usepackage{csquotes}
\renewcommand{\abstractname}{서론}
\renewcommand{\contentsname}{목차}
\usepackage{listings}

\usepackage{color}

\definecolor{dkgreen}{rgb}{0,0.6,0}
\definecolor{gray}{rgb}{0.5,0.5,0.5}
\definecolor{mauve}{rgb}{0.58,0,0.82}

\lstset{frame=tb,
  language=Java,
  aboveskip=3mm,
  belowskip=3mm,
  showstringspaces=false,
  columns=flexible,
  basicstyle={\small\ttfamily},
  numbers=none,
  numberstyle=\tiny\color{gray},
  keywordstyle=\color{blue},
  commentstyle=\color{dkgreen},
  stringstyle=\color{mauve},
  breaklines=true,
  breakatwhitespace=true,
  tabsize=3
}

%%% INFO
\title{\textbf{머신러닝}}
\author{김건호 \\ \phantom{m} \\ \normalsize 서울과학고} % \and
\date{}

%%% DOCUMENT
\begin{document}

\maketitle

\begin{abstract}
테스트
\end{abstract}

\tableofcontents

\section{생각}

\subsection{기초 개념}

Regression과 Classification은 유사하면서도 차이가 있다.
우선 둘 모두 Supervised Learning의 방법이기에 어떤 label이 된 독립 변수와 종속 변수가 있는 데이터 셋을 기준으로
새로운 독립 변수를 받았을 때 나오는 종속 변수를 예측하는 것이다. 
차이점은 종속 변수의 타입이 되는데, 연속적인 값들을 가질 때를 Regression이라 하고 불연속적이고 이산적인 데이터 형태를 가질때를 Classification이라 한다.

이것 이외에도 Clustering%%%%%%%%%%%%%%%%%%%%%%%%%%%%%

\subsection{Iris Classification}

\begin{itemize}
    \item 한 꽃에 해당하는 값 (2차원 배열에서 가로 행): Instance/Observation
    \item 한 성질에 해당하는 값 (2차원 배열에서 세로 열): Attribute/Measurement/Dimension
    \item 예측하려고 하는 값: Target
\end{itemize}

이 중 나중에 모델을 테스트하기 위해서 Training Data와 Testing Data로 나눈다. 
Training Data가 많을 수록 모델에게 좋기는 하지만, 가지고 있는 모든 데이터를 Training Data로 사용하면 이 모델이 정확하게 판단할 수 있는지에 대해서 전혀 근거가 없어지기 위해 한다.

또한 고른 Testing Data가 특별하였을 수도 있기 때문에 모든 간으한 Testing Data에 대해서 

sklearn에는 여러 학습 모델이 있다. 이 중 이번 Iris의 Classification을 위해서는 $k$-NN을 사용한다. 


\subsubsection{$k$-NN}



%\printbibliography

%%% 
\end{document}
